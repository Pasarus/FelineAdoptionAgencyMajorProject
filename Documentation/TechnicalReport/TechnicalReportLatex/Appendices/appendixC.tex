\chapter{Code Examples}

% For some projects, it might be relevant to include some code extracts in an appendix. You are not expected to put all of your code here - the correct place for all of your code is in the technical submission that is made in addition to the Project Report. However, if there are some notable aspects of the code that you discuss, including that in an appendix might be useful to make it easier for your readers to access. 

% As a general guide, if you are discussing short extracts of code then you are advised to include such code in the body of the report. If there is a longer extract that is relevant, then you might include it as shown in the following section. 

% Only include code in the appendix if that code is discussed and referred to in the body of the report. 

% \section{Random Number Generator}

% The Bayes Durham Shuffle ensures that the psuedo random numbers used in the simulation are further shuffled, ensuring minimal correlation between subsequent random outputs \cite{NumericalRecipes}.

% \begin{verbatim}
%  #define IM1 2147483563
%  #define IM2 2147483399
%  #define AM (1.0/IM1)
%  #define IMM1 (IM1-1)
%  #define IA1 40014
%  #define IA2 40692 
%  #define IQ1 53668
%  #define IQ2 52774
%  #define IR1 12211
%  #define IR2 3791
%  #define NTAB 32
%  #define NDIV (1+IMM1/NTAB)
%  #define EPS 1.2e-7
%  #define RNMX (1.0 - EPS)
 
%  double ran2(long *idum)
%  {
%   /*---------------------------------------------------*/
%   /* Minimum Standard Random Number Generator          */
%   /* Taken from Numerical recipies in C                */
%   /* Based on Park and Miller with Bays Durham Shuffle */
%   /* Coupled Schrage methods for extra periodicity     */
%   /* Always call with negative number to initialise    */
%   /*---------------------------------------------------*/	
 
%   int j;
%   long k;
%   static long idum2=123456789;
%   static long iy=0;
%   static long iv[NTAB];
%   double temp;
 
%   if (*idum <=0)
%   {
%      if (-(*idum) < 1)
%      {
%       *idum = 1;
%      }else
%      {
%       *idum = -(*idum);
%      }
%      idum2=(*idum);
%      for (j=NTAB+7;j>=0;j--)
%      {
%       k = (*idum)/IQ1;
%       *idum = IA1 *(*idum-k*IQ1) - IR1*k;
%       if (*idum < 0)
%       {
%          *idum += IM1;
%       }
%       if (j < NTAB)
%       {
%          iv[j] = *idum;
%       }
%      }
%      iy = iv[0];	
%   }
%   k = (*idum)/IQ1;
%   *idum = IA1*(*idum-k*IQ1) - IR1*k;
%   if (*idum < 0)
%   {
%      *idum += IM1;
%   }
%   k = (idum2)/IQ2;
%   idum2 = IA2*(idum2-k*IQ2) - IR2*k;
%   if (idum2 < 0)
%   {
%      idum2 += IM2;
%   }
%   j = iy/NDIV;
%   iy=iv[j] - idum2;
%   iv[j] = *idum;
%   if (iy < 1)
%   {
%      iy += IMM1;
%   }
%   if ((temp=AM*iy) > RNMX)
%   {
%      return RNMX;
%   }else
%   {
%      return temp;	
%   }
%  }
 
% \end{verbatim}

\section{Data Classes} \label {DATACLASSEXAMPLE}

\subsection{User}
\begin{verbatim}
data class User(
    var addressLineOne: String? = "",
    var addressLineTwo: String? = "",
    var addressLineThree: String? = "",
    var county: String? = "",
    var name: String? = "",
    var mobileNumber: String? = "",
    var postCode: String? = "",
    var favouritedCats: List<String>? = ArrayList(),
    var adoptionProcesses: List<DocumentReference>? = ArrayList()
)
\end{verbatim}
    
\subsection{Feedback}
\begin{verbatim}
class Feedback {

    var feedback: String? = null
    var developerReplyRequested: Boolean? = null
    var date: String? = null
    var userDocument: DocumentReference? = null

    constructor() {}
    constructor(
        feedback: String,
        devReplyRequested: Boolean,
        date: String,
        userDocument: DocumentReference
    ) {
        this.feedback = feedback
        this.developerReplyRequested = devReplyRequested
        this.date = date
        this.userDocument = userDocument
    }
}
\end{verbatim}

\subsection{Cat}
\begin{verbatim}
class Cat : Parcelable {

    var catAgeMonths: Int? = 0
    var catBreed: String? = ""
    var catName: String? = ""
    var colour: String? = ""
    var description: String? = ""
    var disabled: Boolean? = false
    var location: String? = ""
    var neutered: Boolean? = false
    var pictureUrl: String? = ""
    var dogs: Boolean? = false
    var indoors: Boolean? = false
    var kids0to4: Boolean? = false
    var kids13to18: Boolean? = false
    var kids5to12: Boolean? = false
    var otherCats: Boolean? = false
    var sex: String? = ""
    var catId: String? = ""

    constructor() : super()  // Needed for Firebase
    constructor(
        catAgeMonths: Int?,
        catBreed: String?,
        catName: String?,
        colour: String?,
        description: String?,
        disabled: Boolean?,
        location: String?,
        neutered: Boolean?,
        pictureUrl: String?,
        dogs: Boolean?,
        indoors: Boolean?,
        kids0to4: Boolean?,
        kids13to18: Boolean?,
        kids5to12: Boolean?,
        otherCats: Boolean?,
        sex: String?,
        catId: String?
    ) {
        this.catAgeMonths = catAgeMonths
        this.catBreed = catBreed
        this.catName = catName
        this.colour = colour
        this.description = description
        this.disabled = disabled
        this.location = location
        this.neutered = neutered
        this.pictureUrl = pictureUrl
        this.dogs = dogs
        this.indoors = indoors
        this.kids0to4 = kids0to4
        this.kids13to18 = kids13to18
        this.kids5to12 = kids5to12
        this.otherCats = otherCats
        this.sex = sex
        this.catId = catId
    }

    constructor(parcel: Parcel) : this() {
        catAgeMonths = parcel.readValue(Int::class.java.classLoader) as? Int
        catBreed = parcel.readString()
        catName = parcel.readString()
        colour = parcel.readString()
        description = parcel.readString()
        disabled = parcel.readValue(Boolean::class.java.classLoader) as? Boolean
        location = parcel.readString()
        neutered = parcel.readValue(Boolean::class.java.classLoader) as? Boolean
        pictureUrl = parcel.readString()
        sex = parcel.readString()
        catId = parcel.readString()
        dogs = stringToBool(parcel.readString() as String)
        indoors = stringToBool(parcel.readString() as String)
        kids0to4 = stringToBool(parcel.readString() as String)
        kids13to18 = stringToBool(parcel.readString() as String)
        kids5to12 = stringToBool(parcel.readString() as String)
        otherCats = stringToBool(parcel.readString() as String)
    }

    override fun writeToParcel(parcel: Parcel, flags: Int) {
        parcel.writeValue(catAgeMonths)
        parcel.writeString(catBreed)
        parcel.writeString(catName)
        parcel.writeString(colour)
        parcel.writeString(description)
        parcel.writeValue(disabled)
        parcel.writeString(location)
        parcel.writeValue(neutered)
        parcel.writeString(pictureUrl)
        parcel.writeString(sex)
        parcel.writeValue(catId)
        parcel.writeString(boolToString(dogs!!))
        parcel.writeString(boolToString(indoors!!))
        parcel.writeString(boolToString(kids0to4!!))
        parcel.writeString(boolToString(kids13to18!!))
        parcel.writeString(boolToString(kids5to12!!))
        parcel.writeString(boolToString(otherCats!!))
    }

    override fun describeContents(): Int {
        return 0
    }

    companion object CREATOR : Parcelable.Creator<Cat> {
        override fun createFromParcel(parcel: Parcel): Cat {
            return Cat(parcel)
        }

        override fun newArray(size: Int): Array<Cat?> {
            return arrayOfNulls(size)
        }
    }
}
\end{verbatim}

\subsection{AdoptionProcess}
\begin{verbatim}
class AdoptionProcess : Parcelable {
    var cat: DocumentReference? = null
    var status: MutableMap<String, Any>? = HashMap()
    var user: DocumentReference? = null

    constructor() : super()  // Needed for Firebase
    constructor(status: Map<String, Any>, cat: Cat) : this() {
        this.status = status.toMutableMap()
        this.cat = FirebaseFirestore.getInstance().collection("cats")
            .document("cat" + cat.catId)

        this.user = FirebaseFirestore.getInstance().collection("users").document(
            DataService.INSTANCE.user!!.uid
        )
    }

    constructor(parcel: Parcel) : this() {
        status!!["pending"] = stringToBool(parcel.readString() as String)
        status!!["pendingReason"] = parcel.readString() as String
        status!!["accepted"] = stringToBool(parcel.readString() as String)
        status!!["rejected"] = stringToBool(parcel.readString() as String)
        status!!["rejectedReason"] = parcel.readString() as String
        user = FirebaseFirestore.getInstance().document(parcel.readString() as String)
        cat = FirebaseFirestore.getInstance().document(parcel.readString() as String)
    }

    override fun writeToParcel(dest: Parcel?, flags: Int) {
        if (dest != null){
            val pending = status!!["pending"] as Boolean
            val pendingReason = status!!["pendingReason"] as String
            val accepted = status!!["accepted"] as Boolean
            val rejected = status!!["rejected"] as Boolean
            val rejectedReason = status!!["rejectedReason"] as String
            val user = this.user.toString()
            val cat = this.cat.toString()
            dest.writeString(boolToString(pending))
            dest.writeString(pendingReason)
            dest.writeString(boolToString(accepted))
            dest.writeString(boolToString(rejected))
            dest.writeString(rejectedReason)
            dest.writeString(user)
            dest.writeString(cat)
        }
    }

    override fun describeContents(): Int {
        return 0
    }

    companion object CREATOR : Parcelable.Creator<AdoptionProcess> {
        override fun createFromParcel(parcel: Parcel): AdoptionProcess {
            return AdoptionProcess(parcel)
        }

        override fun newArray(size: Int): Array<AdoptionProcess?> {
            return arrayOfNulls(size)
        }
    }
}
\end{verbatim}

\section{Server Scripts}

\subsection{Create Cats Script}\label{CREATECATSCRIPT}
\begin{verbatim}
import os
import firebase_admin
from google.cloud import firestore
from json import dump, load
from Cat import Cat
from random import randint


"""
This script was designed with the intention of generating cats for my MMP 
project continuing the Feline Adoption Agency App for Android. With that
in mind it utilizes a backend database from Google Firestore. For 
someone to use this script they must change 
GOOGLE_SERVICES_ACCOUNT_KEY_LOCATION to reflect their own database.
"""


GOOGLE_SERVICES_ACCOUNT_KEY_LOCATION = ""
FILE_PATH_TO_COPY_OF_CATS = "CatCopies/copy.json"
CAT_DETAILS_DIR = "CatDetails"
CAT_NAMES_FILE = "CatNames.json"
CAT_BREEDS_FILE = "CatBreeds.json"
CAT_DESCRIPTIONS_FILE = "CatDescriptions.json"
CAT_COLOURS_FILE = "CatColours.json"
CAT_LOCATIONS_FILE = "CatLocations.json"
CAT_PICTURES_URLS_FILE = "CatPictureUrls.json"
MAX_CAT_AGE_IN_MONTHS = 100
MIN_CAT_AGE_IN_MONTHS = 0
CAT_ID_TO_START_FROM = 10
CAT_ID_TO_END_AT = 100


def load_cat_breeds():
    with open(os.path.join(CAT_DETAILS_DIR, CAT_BREEDS_FILE)) as breeds_file:
        return load(breeds_file)


def load_cat_descriptions():
    with open(os.path.join(CAT_DETAILS_DIR, CAT_DESCRIPTIONS_FILE)) as \
    descriptions_file:
        return load(descriptions_file)


def load_cat_names():
    with open(os.path.join(CAT_DETAILS_DIR, CAT_NAMES_FILE)) as names_file:
        return load(names_file)


def load_colours():
    with open(os.path.join(CAT_DETAILS_DIR, CAT_COLOURS_FILE)) as colours_file:
        return load(colours_file)


def load_locations():
    with open(os.path.join(CAT_DETAILS_DIR, CAT_LOCATIONS_FILE)) as \
    locations_file:
        return load(locations_file)


def load_picture_urls():
    with open(os.path.join(CAT_DETAILS_DIR, CAT_PICTURES_URLS_FILE)) as \
    url_file:
        return load(url_file)


def load_cats_details():
    return load_cat_names(), load_cat_breeds(), \
    load_cat_descriptions(), load_colours(), \
    load_locations(), load_picture_urls()


def generate_cats():
    cats_list = []

    cat_names, cat_breeds, cat_descriptions, colours, locations, picture_urls \
    = load_cats_details()

    true_or_false = [True, False]
    male_or_female = [u"Male", u"Female"]

    for ii in range(CAT_ID_TO_START_FROM, CAT_ID_TO_END_AT):
        children = true_or_false[randint(0, 1)]
        cats_list.append(Cat(
            cat_name=cat_names[randint(0, len(cat_names) - 1)],
            cat_breed=cat_breeds[randint(0, len(cat_breeds) - 1)],
            description=cat_descriptions[randint(0,
                len(cat_descriptions) - 1)],
            cat_age_months=randint(
                a=MIN_CAT_AGE_IN_MONTHS,
                b=MAX_CAT_AGE_IN_MONTHS),
            cat_id=str(ii),
            colour=colours[randint(0, len(colours) - 1)],
            disabled=true_or_false[randint(0, 1)],
            neutered=true_or_false[randint(0, 1)],
            dogs=true_or_false[randint(0, 1)],
            indoors=true_or_false[randint(0, 1)],
            kids0to4=children,
            kids5to12=children,
            kids13to18=children,
            location=locations[randint(0, len(locations) - 1)],
            other_cats=true_or_false[randint(0, 1)],
            picture_url=picture_urls[randint(0, len(picture_urls) - 1)],
            sex=male_or_female[randint(0, 1)]
        ).to_dict())

    return cats_list


def upload_cats(cats_list):
    # Ensure that firestore is initialized:
    os.environ["GOOGLE_APPLICATION_CREDENTIALS"] = \ 
    GOOGLE_SERVICES_ACCOUNT_KEY_LOCATION
    firebase_admin.initialize_app()

    db = firestore.Client()

    for ii in range(CAT_ID_TO_START_FROM, CAT_ID_TO_END_AT):
        # Put each of the cats online in the correct file.
        db.collection("cats").document("cat"+str(ii)) \
        .set(cats_list[ii-CAT_ID_TO_START_FROM])


# This is the actual script that is ran, it subsequently calls 
# all the functions
cats = generate_cats()
with open(FILE_PATH_TO_COPY_OF_CATS, "w+") as file:
    dump(cats, file)
upload_cats(cats)
\end{verbatim}

\subsection{Remove Cats Script}\label{REMOVECATSSCRIPT}

\begin{verbatim}
"""
This script was made when I realised I made a mistake in my cat generation
script for all cat ids between 10 and 99. So this is designed at removing 
those cats. With that in mind it utilizes a backend database from Google Firestore.
For someone to use this script they must change
GOOGLE_SERVICES_ACCOUNT_KEY_LOCATION to reflect their own database.
"""
import os

import firebase_admin
from google.cloud import firestore

GOOGLE_SERVICES_ACCOUNT_KEY_LOCATION = ""
CAT_ID_TO_START_FROM = 10
CAT_ID_TO_END_AT = 100

os.environ["GOOGLE_APPLICATION_CREDENTIALS"] = \
GOOGLE_SERVICES_ACCOUNT_KEY_LOCATION
firebase_admin.initialize_app()

db = firestore.Client()

for ii in range(CAT_ID_TO_START_FROM, CAT_ID_TO_END_AT):
    db.collection("cats").document("cat" + str(ii)).delete()
\end{verbatim}

\subsection{Cat Class}\label{CATCLASSSCRIPT}
\begin{verbatim}
class Cat(object):
    def __init__(self, cat_age_months, cat_breed, cat_name, colour, 
                    description, disabled, location, neutered,
                    picture_url, dogs, indoors, kids0to4, kids13to18,
                    kids5to12, other_cats, sex, cat_id):
        self.cat_age_months = cat_age_months
        self.cat_breed = cat_breed
        self.cat_name = cat_name
        self.colour = colour
        self.description = description
        self.disabled = disabled
        self.location = location
        self.neutered = neutered
        self.picture_url = picture_url
        self.dogs = dogs
        self.indoors = indoors
        self.kids0to4 = kids0to4
        self.kids13to18 = kids13to18
        self.kids5to12 = kids5to12
        self.other_cats = other_cats
        self.sex = sex
        self.cat_id = cat_id

    def to_dict(self):
        return {
            u"catAgeMonths": self.cat_age_months,
            u"catBreed": self.cat_breed,
            u"catName": self.cat_name,
            u"colour": self.colour,
            u"description": self.description,
            u"disabled": self.disabled,
            u"location": self.location,
            u"neutered": self.neutered,
            u"pictureUrl": self.picture_url,
            u"dogs": self.dogs,
            u"indoors": self.indoors,
            u"kids0to4": self.kids0to4,
            u"kids13to18": self.kids13to18,
            u"kids5to12": self.kids5to12,
            u"otherCats": self.other_cats,
            u"sex": self.sex,
            u"catId": self.cat_id
        }
\end{verbatim}

\section{NavigationUI Navigation Graph} \label{NAVIGATIONGRAPH}
\begin{verbatim}
    <?xml version="1.0" encoding="utf-8"?>

<!-- Copyright 2020 Samuel Jones

   Licensed under the Apache License, Version 2.0 (the "License");
   you may not use this file except in compliance with the License.
   You may obtain a copy of the License at

       http://www.apache.org/licenses/LICENSE-2.0

   Unless required by applicable law or agreed to in writing, software
   distributed under the License is distributed on an "AS IS" BASIS,
   WITHOUT WARRANTIES OR CONDITIONS OF ANY KIND, either express or implied.
   See the License for the specific language governing permissions and
   limitations under the License. -->

<navigation xmlns:android="http://schemas.android.com/apk/res/android"
    xmlns:app="http://schemas.android.com/apk/res-auto"
    xmlns:tools="http://schemas.android.com/tools"
    android:id="@+id/nav_graph"
    app:startDestination="@id/homeFragment">

    <fragment
        android:id="@+id/settingsFragment"
        android:name="uk.ac.aber.dcs.mmp.faa.ui.settings.SettingsFragment"
        android:label="Settings"
        tools:layout="@layout/settings_fragment" />
    <fragment
        android:id="@+id/myAccountFragment"
        android:name="uk.ac.aber.dcs.mmp.faa.ui.account.MyAccountFragment"
        android:label="Account Details"
        tools:layout="@layout/my_account_fragment" />
    <fragment
        android:id="@+id/helpFragment"
        android:name="uk.ac.aber.dcs.mmp.faa.ui.help.HelpFragment"
        android:label="Help"
        tools:layout="@layout/help_fragment" />
    <fragment
        android:id="@+id/aboutFragment"
        android:name="uk.ac.aber.dcs.mmp.faa.ui.about.AboutFragment"
        android:label="About"
        tools:layout="@layout/about_fragment" />
    <fragment
        android:id="@+id/feedbackFragment"
        android:name="uk.ac.aber.dcs.mmp.faa.ui.feedback.FeedbackFragment"
        android:label="Feedback"
        tools:layout="@layout/feedback_fragment" />
    <fragment
        android:id="@+id/homeFragment"
        android:name="uk.ac.aber.dcs.mmp.faa.ui.home.HomeFragment"
        android:label="Home"
        tools:layout="@layout/home_fragment" >
        <action
            android:id="@+id/action_homeFragment_to_settingsFragment"
            app:destination="@id/settingsFragment" />
        <action
            android:id="@+id/action_homeFragment_to_myAccountFragment"
            app:destination="@id/myAccountFragment" />
        <action
            android:id="@+id/action_homeFragment_to_helpFragment"
            app:destination="@id/helpFragment" />
        <action
            android:id="@+id/action_homeFragment_to_aboutFragment"
            app:destination="@id/aboutFragment" />
        <action
            android:id="@+id/action_homeFragment_to_feedbackFragment"
            app:destination="@id/feedbackFragment" />
        <action
            android:id="@+id/action_homeFragment_to_savedFragment"
            app:destination="@id/savedFragment" />
        <action
            android:id="@+id/action_homeFragment_to_findCatFragment"
            app:destination="@id/findCatFragment" />
        <action
            android:id=
            "@+id/action_homeFragment_to_adoptionStatusInfoViewFragment"
            app:destination="@id/adoptionStatusInfoViewFragment" />
        <action
            android:id="@+id/action_homeFragment_to_catCardInfoFragment"
            app:destination="@id/catCardInfoFragment" />
        <action
            android:id="@+id/action_homeFragment_to_adoptionForm"
            app:destination="@id/adoptionForm" />
        <action
            android:id="@+id/action_homeFragment_to_adoptionFormConfirmation"
            app:destination="@id/adoptionFormConfirmation" />
        <action
            android:id="@+id/action_homeFragment_to_cancel_adoption_dialog"
            app:destination="@id/cancel_adoption_dialog" />
    </fragment>
    <fragment
        android:id="@+id/savedFragment"
        android:name="uk.ac.aber.dcs.mmp.faa.ui.saved.SavedFragment"
        android:label="Saved Cats"
        tools:layout="@layout/saved_fragment" />
    <fragment
        android:id="@+id/findCatFragment"
        android:name="uk.ac.aber.dcs.mmp.faa.ui.findcat.FindCatFragment"
        android:label="Find A Cat Tool"
        tools:layout="@layout/find_cat_fragment" />
    <fragment
        android:id="@+id/adoptionStatusInfoViewFragment"
    android:name="uk.ac.aber.dcs.mmp.faa.ui.adoption.
                                    AdoptionStatusInfoViewFragment"
        android:label="Adoption Status Info"
        tools:layout="@layout/adoption_status_info_view_fragment" />
    <fragment
        android:id="@+id/catCardInfoFragment"
        android:name="uk.ac.aber.dcs.mmp.faa.ui.catinfo.CatCardInfoFragment"
        android:label="More Info"
        tools:layout="@layout/cat_card_info_fragment" />
    <fragment
        android:id="@+id/adoptionForm"
        android:name="uk.ac.aber.dcs.mmp.faa.ui.adoption.AdoptionForm"
        android:label="Adoption Form"
        tools:layout="@layout/fragment_adoption_form" />
    <fragment
        android:id="@+id/adoptionFormConfirmation"
        android:name="uk.ac.aber.dcs.mmp.faa.ui.adoption.
                                AdoptionFormConfirmation"
        android:label="Adoption Confirmation"
        tools:layout="@layout/fragment_adoption_form_confirmation" />
    <dialog
        android:id="@+id/cancel_adoption_dialog"
        android:name="uk.ac.aber.dcs.mmp.faa.ui.adoption.CancelAdoptionDialog"
        android:label="Adoption Cancellation confirmation"
        tools:layout="@layout/fragment_cancel_adoption_dialog" />
</navigation>
\end{verbatim}

\section{Firebase Filter Implementation} \label{FIRESTOREQUERYIMPLEMENTATION}
\begin{verbatim}
var query: Query = FirebaseFirestore.getInstance().collection("cats")

val location = map["location"]
if (location != "") {
    query = query.whereEqualTo("location", location)
}

val families = map["families"]
if (families != "") {
    if (families == "Children") {
        query = query.whereEqualTo("kids0to4", true)
        query = query.whereEqualTo("kids5to12", true)
        query = query.whereEqualTo("kids13to18", true)
    } else if (families == "No Children") {
        query = query.whereEqualTo("kids0to4", false)
        query = query.whereEqualTo("kids5to12", false)
        query = query.whereEqualTo("kids13to18", false)
    }
}

val otherCats = map["otherCats"]
if (otherCats != "") {
    if (otherCats == "Other cats welcome") {
        query = query.whereEqualTo("otherCats", true)
    } else if (otherCats == "No other cats") {
        query = query.whereEqualTo("otherCats", false)
    }
}

val dogs = map["dogs"]
if (dogs != "") {
    if (dogs == "Happy with Dogs") {
        query = query.whereEqualTo("dogs", true)
    } else if (dogs == "Unhappy with Dogs") {
        query = query.whereEqualTo("dogs", false)
    }
}

val indoors = map["indoorsOnly"]
if (indoors != "") {
    if (indoors == "Indoors only") {
        query = query.whereEqualTo("indoors", true)
    } else if (indoors == "Needs outside access") {
        query = query.whereEqualTo("indoors", false)
    }
}

val sortBy = map["sortBy"]
var orderByString = ""
when (sortBy) {
    "Name" -> {
        orderByString = "catName"
    }
    "Age" -> {
        orderByString = "catAgeMonths"
    }
    "Recently Listed" -> {
        orderByString = "catId"
    }
}

query = if (map["sortByOrdering"] == "Ascending"){
    query.orderBy(orderByString, Query.Direction.ASCENDING)
} else {
    query.orderBy(orderByString, Query.Direction.DESCENDING)
}

val name = map["name"]
if (name != "") {
    query = query.startAt(name).endAt(name + "\uf8ff")
}
\end{verbatim}

\section{Testing Examples} 
The following are examples given for testing in the project.

\subsection{Unit Test example} \label{Unit Test Example}
\begin{verbatim}
import junit.framework.Assert.assertEquals
import org.junit.Before
import org.junit.Test
import uk.ac.aber.dcs.mmp.faa.utils.ObserverOfStringSet
import uk.ac.aber.dcs.mmp.faa.utils.SimpleObservableStringSet

class SimpleObservableStringSetTests : ObserverOfStringSet {
    private var addCalledCount: Int = 0
    private var removeCalledCount: Int = 0
    private lateinit var testSet: SimpleObservableStringSet

    @Before
    fun setUp() {
        addCalledCount = 0
        removeCalledCount = 0
        testSet = SimpleObservableStringSet()
        testSet.addObserver(this)
    }

    override fun onObservedAdd(e: String) {
        addCalledCount += 1
    }

    override fun onObservedRemove(e: String) {
        removeCalledCount += 1
    }

    @Test
    fun test_notifiesObserversOnAdd() {
        testSet.add("test1")
        assertEquals(addCalledCount, 1)
    }

    @Test
    fun test_notifiesObserversOnAddOfDuplicate() {
        val test1 = "test1"
        testSet.add(test1)
        assertEquals(addCalledCount, 1)
        testSet.add(test1)
        assertEquals(addCalledCount, 2)
    }

    @Test
    fun test_notifiesObserversOnRemove() {
        val test1 = "test1"
        testSet.add(test1)
        testSet.remove(test1)
        assertEquals(addCalledCount, 1)
        assertEquals(removeCalledCount, 1)
    }

    @Test
    fun test_notifiesObserversOnRemoveOfNoneExistentString() {
        testSet.remove("test1")
        assertEquals(removeCalledCount, 1)
    }

    @Test
    fun test_notifiesMultipleTimesForMultipleAdds() {
        testSet.add("test1")
        assertEquals(addCalledCount, 1)
        testSet.add("test2")
        assertEquals(addCalledCount, 2)
        testSet.add("test3")
        assertEquals(addCalledCount, 3)
    }

    @Test
    fun test_notifiesMultipleTimesForMultipleRemoves() {
        val test1 = "test1"
        val test2 = "test2"
        val test3 = "test3"
        testSet.add(test1)
        assertEquals(addCalledCount, 1)
        testSet.add(test2)
        assertEquals(addCalledCount, 2)
        testSet.add(test3)
        assertEquals(addCalledCount, 3)
        testSet.remove(test1)
        assertEquals(removeCalledCount, 1)
        testSet.remove(test2)
        assertEquals(removeCalledCount, 2)
        testSet.remove(test3)
        assertEquals(removeCalledCount, 3)
    }

    @Test
    fun test_notifiesOncePerItemAddedWhenAddingMultipleItemsWithAddAll() {
        testSet.addAll(setOf("test1", "test2", "test3"))
        assertEquals(addCalledCount, 3)
    }

    @Test
    fun test_notifiesOncePerItemClearedWhenUsingClear() {
        testSet.addAll(setOf("test1", "test2", "test3"))
        assertEquals(addCalledCount, 3)

        testSet.clear()
        assertEquals(removeCalledCount, 3)
    }

    @Test
    fun test_notifiesOncePerItemClearedWhenRemovingMultipleItemsWithRemoveAll() {
        testSet.addAll(setOf("test1", "test2", "test3"))
        assertEquals(addCalledCount, 3)

        testSet.removeAll(setOf("test1", "test2", "test3"))
        assertEquals(removeCalledCount, 3)
    }

    // Tests below this point are to ensure that basic set functionality
    // is still viable via this class
    @Test
    fun test_setSizeOption() {
        for (i in 1..10) {
            testSet.add("" + i)
            assertEquals(testSet.size, i)
        }
    }

    @Test
    fun test_setUniquenessIsGuarenteed() {
        val test1 = "test1"
        testSet.add(test1)
        assertEquals(testSet.size, 1)
        testSet.add(test1)
        assertEquals(testSet.size, 1)

        // Ensure the value in the set is test1
        for (wannabeTest1 in testSet) {
            assertEquals(wannabeTest1, test1)
        }
    }

    @Test
    fun test_setRetrievalViaIteratorStillWorks() {
        val test1 = "test1"
        testSet.add(test1)
        for (wannabeTest1 in testSet) {
            assertEquals(wannabeTest1, test1)
        }
    }

    @Test
    fun test_ensureUpdateOccursWithBlankFunction() {
        testSet.updateObserversAddBlank()
        assertEquals(addCalledCount, 1)
    }
}
\end{verbatim}

\subsection{UI Instrumentation Test example} \label{UIINSTTESTEXAMPLE}
\begin{verbatim}
import android.view.View
import android.view.ViewGroup
import androidx.test.espresso.Espresso.onView
import androidx.test.espresso.action.ViewActions.click
import androidx.test.espresso.assertion.ViewAssertions.matches
import androidx.test.espresso.matcher.ViewMatchers.*
import androidx.test.filters.LargeTest
import androidx.test.internal.runner.junit4.AndroidJUnit4ClassRunner
import androidx.test.rule.ActivityTestRule
import org.hamcrest.Description
import org.hamcrest.Matcher
import org.hamcrest.Matchers.`is`
import org.hamcrest.Matchers.allOf
import org.hamcrest.TypeSafeMatcher
import org.hamcrest.core.IsInstanceOf
import org.junit.Rule
import org.junit.Test
import org.junit.runner.RunWith
import uk.ac.aber.dcs.mmp.faa.R

@LargeTest
@RunWith(AndroidJUnit4ClassRunner::class)
class AboutFragmentContents {

    @Rule
    @JvmField
    var mActivityTestRule = ActivityTestRule(MainActivity::class.java)

    @Test
    fun aboutFragmentContents() {
        val appCompatImageButton = onView(
            allOf(
                withContentDescription("Open navigation drawer"),
                childAtPosition(
                    allOf(
                        withId(R.id.toolbar),
                        childAtPosition(
                            withClassName(`is`(
                            "androidx.constraintlayout.widget.ConstraintLayout")),
                            0
                        )
                    ),
                    1
                ),
                isDisplayed()
            )
        )
        appCompatImageButton.perform(click())

        val appCompatTextView = onView(
            allOf(
                withId(R.id.navDrawAbout), withText("About"),
                childAtPosition(
                    childAtPosition(
                        withId(R.id.navDrawerNavView),
                        1
                    ),
                    5
                ),
                isDisplayed()
            )
        )
        appCompatTextView.perform(click())

        val imageView = onView(
            allOf(
                withContentDescription("Cat Logo Material Design Icon"),
                childAtPosition(
                    childAtPosition(
                        IsInstanceOf.instanceOf(
                        android.widget.LinearLayout::class.java),
                        6
                    ),
                    0
                ),
                isDisplayed()
            )
        )
        imageView.check(matches(isDisplayed()))

        val textView = onView(
            allOf(
                withText("The following icons, were made under SIL
                Open Font License 1.1 by @Templarian on Twitter:"),
                childAtPosition(
                    childAtPosition(
                        withId(R.id.navHostFragment),
                        0
                    ),
                    5
                ),
                isDisplayed()
            )
        )
        textView.check(matches(withText("The following icons, were
        made under SIL Open Font License 1.1 by @Templarian on 
        Twitter:")))

        val textView2 = onView(
            allOf(
                withText("The icons created by google as part of 
                Material Design are distributed under an Apache 2.0 License
                (http://www.apache.org/licenses/LICENSE-2.0)."),
                childAtPosition(
                    childAtPosition(
                        withId(R.id.navHostFragment),
                        0
                    ),
                    4
                ),
                isDisplayed()
            )
        )
        textView2.check(matches(withText("The icons created by google as
        part of Material Design are distributed under an Apache 2.0 License
        (http://www.apache.org/licenses/LICENSE-2.0).")))

        val textView3 = onView(
            allOf(
                withText("As part of this application certain icons have
                been used, these Material Design based icon are created 
                by the community and are distrubuted under SIL Open Font 
                License 1.1 (https://opensource.org/licenses/OFL-1.1)."),
                childAtPosition(
                    childAtPosition(
                        withId(R.id.navHostFragment),
                        0
                    ),
                    3
                ),
                isDisplayed()
            )
        )
        textView3.check(matches(withText("As part of this application 
        certain icons have been used, these Material Design based icon
        are created by the community and are distrubuted under SIL Open
        Font License 1.1 (https://opensource.org/licenses/OFL-1.1).")))

        val textView4 = onView(
            allOf(
                withText("It is distributed open source at 
                https://github.com/Pasarus/FelineAdoptionAgencyMajorProject"),
                childAtPosition(
                    childAtPosition(
                        withId(R.id.navHostFragment),
                        0
                    ),
                    2
                ),
                isDisplayed()
            )
        )
        textView4.check(matches(withText("It is distributed open source 
        at https://github.com/Pasarus/FelineAdoptionAgencyMajorProject")))

        val textView5 = onView(
            allOf(
                withText("Whilst use of this applications source code is
                free to everyone, the explicit copyright is owned by the 
                sole author Samuel Jones (samjones714@gmail.com)"),
                childAtPosition(
                    childAtPosition(
                        withId(R.id.navHostFragment),
                        0
                    ),
                    1
                ),
                isDisplayed()
            )
        )
        textView5.check(matches(withText("Whilst use of this applications
        source code is free to everyone, the explicit copyright is owned 
        by the sole author Samuel Jones (samjones714@gmail.com)")))

        val textView6 = onView(
            allOf(
                withText("This application is meant as a demonstration of 
                how the adoption of a cat could be fully integrated into 
                an app via a backend database and user verification system.
                This mobile application was created under an Apache 2.0 
                License (https://www.apache.org/licenses/LICENSE-2.0)"),
                childAtPosition(
                    childAtPosition(
                        withId(R.id.navHostFragment),
                        0
                    ),
                    0
                ),
                isDisplayed()
            )
        )
        textView6.check(matches(withText("This application is meant as a
        demonstration of how the adoption of a cat could be fully integrated
        into an app via a backend database and user verification system. 
        This mobile application was created under an Apache 2.0 License 
        (https://www.apache.org/licenses/LICENSE-2.0)")))
    }

    private fun childAtPosition(
        parentMatcher: Matcher<View>, position: Int
    ): Matcher<View> {

        return object : TypeSafeMatcher<View>() {
            override fun describeTo(description: Description) {
                description.appendText("Child at position $position in parent ")
                parentMatcher.describeTo(description)
            }

            public override fun matchesSafely(view: View): Boolean {
                val parent = view.parent
                return parent is ViewGroup && parentMatcher.matches(parent)
                        && view == parent.getChildAt(position)
            }
        }
    }
}
\end{verbatim}