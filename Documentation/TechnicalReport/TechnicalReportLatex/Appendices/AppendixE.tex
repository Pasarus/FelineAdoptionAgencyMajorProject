\chapter{Sprint Retrospectives} \label {SRINTRETROSPECTIVES}

\section{Sprint 6 (09/04/2020 - 22/04/2020) Retrospective}
This is the final sprint that has been completed for this project, as no more engineering work will be completed for this project. The final bugs have been fixed before the report is written, some may be completed after to ensure that the final demonstration goes to plan. The Sprint 6 Kanban board is going to be continued for the remainder of the project. 

This sprint a few bug fixes were completed and quite a few things were set up properly for the report. The Report has been the absolute focus of almost all work time during this sprint, and 2 chapters have been close to completion, with another chapter being started. I am on track to finish the report on time but more time could be spent focused on it in future so I will endeavour to do so.

\section{Sprint 5 (26/03/2020 - 08/04/2020) Retrospective}
Sprint 5 was the sprinting sprint, where a large portion of work was completed in a shorter amount of time than previous, this was in an effort to keep up with my overall goals in light of the recent global pandemic, and my life being upheaved. In this time I managed to complete almost double the amount of work normally completed in this time frame, this was achieved by working more hours than is healthy and is in violation of the agile principles. Due to the nature of the project having a hard deadline, I found it made most sense to have a cram near a soft deadline, then just not have a finished product.

At the end of this iteration, there is a significant piece of progress, the application actually feels finished and somewhat polished, whilst some areas could do with improvement, I believe it needs some serious consideration from a professional UX developer to produce an improved design/implementation in most areas.

\section{Sprint 4 (12/03/2020 - 25/03/2020) Retrospective}
This sprint was quite successful, a large amount of work was completed. Unfortunately, not all work was able to be completed on time, due to current global circumstances with the ongoing COVID-19 pandemic, I was forced to reduce time spent working in favour of arranging for family and personal food and supplies as well as ensuring the home was ready for many people working from home.

With all that said, I was able to produce a piece of software with some significant use to the user. The product is now heavily featured in comparison to previous sprints and certainly feels more "finished", however, there is still some ways to go so I am ramping up efforts to complete the work by the end of sprint 5.

\section{Sprint 3 (27/02/2020 - 11/03/2020) Retrospective}
Sprint 3 was completed on time and all objectives were not strictly met. This is due to other work commitments, as such I will attempt to incorporate other work commitments into this project's plan and thus allow me to plan out my work more effectively.

This sprint I was able to produce a piece of software that actually has some use to the user, i.e. the back-end database has been successfully connected up to the application. The RESTfulAPI is now provided by Google Firebase Firestore, opposed to python Flask as previously decided. The reason for the change in API is because of the ease of integrate-ability of Firebase due to open source libraries opposed to using Flask. I believe that you shouldn't aim to reinvent the wheel.

Also, alongside Firestore picture hosting and Authentication is achieved using Google Firebase. Users can at the moment login to their own accounts with Google and a email / password specific login.

\section{Sprint 2 (13/02/2020 - 26/02/2020) Retrospective}
2nd Sprint is the first sprint in which an actual product was produced. The sprint was poorly planned as not enough work was allocated for current work speeds. In the future, allocation of further tasks should be done, or at the very least further exploration of my capabilities in a 2-week iteration.

This sprint yielded a minor amount of software and allowed me to work on the kotlin code, XML, and some minor networking to allow for the choice of back-end server API / languages. I have decided on Python Flask for my RESTfulAPI.

\section{Sprint 1 (29/01/2020 - 12/02/2020) Retrospective}

The first sprint was completed on time, Icebox issues were attempted and ended up being assigned to the next Sprint as aimed targets. The assigned work was achieved in almost the exact amount of time and current estimations assume that continuing at this rate of estimation will be a successful methodology. 

First sprint dedicated to planning and process development was very successful, however no application development has occurred yet, therefore one tenet of Scrumban has failed and needs to be addressed later, no useful software has been made this sprint.